\documentclass{article}
\usepackage{blindtext}
\usepackage{booktabs}
\usepackage[margin=0.25in]{geometry}
\usepackage{subcaption}
\usepackage{graphicx}
\usepackage{caption}
\usepackage{hyperref}
\usepackage{pdflscape}
\usepackage{tikz}


\title{Descriptive Tables and Figures for Municipality Proliferation}

\begin{document}
\maketitle
\tableofcontents
{\footnotesize 
\listoffigures
\listoftables}
\clearpage

\section{Distributions}
\begin{figure}
	\centering
	\includegraphics[width=.8\textwidth]{figures/distributions/GM_full.png}
\end{figure}
\clearpage
\begin{figure}
	\centering
	\includegraphics[width=.8\textwidth]{figures/distributions/GM_hat_full.png}
\end{figure}
\clearpage
\begin{figure}
	\centering
	\includegraphics[width=.8\textwidth]{figures/distributions/GM_raw_full.png}
\end{figure}
\clearpage
\begin{figure}
	\centering
	\includegraphics[width=.8\textwidth]{figures/distributions/GM_hat_raw_full.png}
\end{figure}
\clearpage
\begin{figure}
	\centering
	\includegraphics[width=.8\textwidth]{figures/distributions/cgoodman_1940.png}
	\caption{1940-50}
\end{figure}
\clearpage
\begin{figure}
	\centering
	\includegraphics[width=.8\textwidth]{figures/distributions/cgoodman_1950.png}
	\caption{1950-60}
\end{figure}
\clearpage
\begin{figure}
	\centering
	\includegraphics[width=.8\textwidth]{figures/distributions/cgoodman_1960.png}
	\caption{1960-70}
\end{figure}
\clearpage

\section{Jasper vs. Lake Counties}
\clearpage
\begin{figure}
	\centering
	\includegraphics[width=.8\textwidth]{figures/descriptive/comparison.png}
\end{figure}
\clearpage
\begin{figure}
	\centering
	\includegraphics[width=.8\textwidth]{figures/descriptive/comparison_hat.png}
\end{figure}
\clearpage
\section{Goldsmith-Pinkham Table}
\begin{landscape}
{
\def\sym#1{\ifmmode^{#1}\else\(^{#1}\)\fi}
\begin{tabular}{l*{8}{c}}
\toprule
                &\multicolumn{6}{c}{County Counts Outcomes}                                               &\multicolumn{1}{c}{Directory Survey Outcomes}&\multicolumn{1}{c}{Instrument}\\\cmidrule(lr){2-7}\cmidrule(lr){8-8}\cmidrule(lr){9-9}
                &\multicolumn{1}{c}{all\_local}&\multicolumn{1}{c}{all\_local\_nosch}&\multicolumn{1}{c}{gen\_subcounty}&\multicolumn{1}{c}{gen\_muni}&\multicolumn{1}{c}{schdist\_ind}&\multicolumn{1}{c}{spdist}&\multicolumn{1}{c}{ngov3}&\multicolumn{1}{c}{GM\_hat}\\
\midrule
base\_outcome    &      -0.22***&       0.10***&       0.02*  &       0.05***&      -0.48***&       0.05   &       0.05***&              \\
                &     (0.03)   &     (0.02)   &     (0.01)   &     (0.02)   &     (0.03)   &     (0.04)   &     (0.01)   &              \\
\addlinespace
mfg\_lfshare     &       0.23** &       0.03   &       0.01*  &       0.01*  &       0.03   &       0.02   &       0.01** &       0.57***\\
                &     (0.09)   &     (0.03)   &     (0.01)   &     (0.01)   &     (0.06)   &     (0.03)   &     (0.00)   &     (0.08)   \\
\addlinespace
blackmig3539\_share&      17.49***&       0.01   &       1.59   &       0.75   &       8.53** &      -1.15   &       1.17   &      94.92***\\
                &     (6.40)   &     (3.14)   &     (1.35)   &     (1.24)   &     (4.31)   &     (3.16)   &     (0.78)   &    (15.44)   \\
\addlinespace
countypop       &       0.00   &       0.00   &       0.00   &       0.00   &       0.00*  &       0.00   &       0.00** &       0.00***\\
                &     (0.00)   &     (0.00)   &     (0.00)   &     (0.00)   &     (0.00)   &     (0.00)   &     (0.00)   &     (0.00)   \\
\addlinespace
Midwest         &     -10.25***&      -0.77   &       0.79***&       0.79***&      -1.06   &      -2.21*  &       1.49***&       4.67** \\
                &     (2.29)   &     (1.01)   &     (0.19)   &     (0.16)   &     (1.47)   &     (1.17)   &     (0.20)   &     (2.13)   \\
\addlinespace
South           &     -12.67***&      -0.46   &       0.05   &       0.19   &      -5.39*  &      -3.61*  &       0.64** &     -14.14   \\
                &     (4.13)   &     (1.81)   &     (0.53)   &     (0.45)   &     (2.92)   &     (1.96)   &     (0.32)   &     (8.81)   \\
\addlinespace
West            &       6.07** &       2.72   &       1.54***&       1.53***&       4.08** &      -0.05   &       2.13***&      -6.59** \\
                &     (2.86)   &     (1.76)   &     (0.48)   &     (0.34)   &     (1.84)   &     (1.44)   &     (0.33)   &     (3.25)   \\
\addlinespace
1940            &       0.00   &       0.00   &       0.00   &       0.00   &       0.00   &       0.00   &       0.00   &       0.00   \\
                &        (.)   &        (.)   &        (.)   &        (.)   &        (.)   &        (.)   &        (.)   &        (.)   \\
\addlinespace
1950            &       2.36   &       3.93***&       0.50** &       0.58** &      -7.52***&       3.62***&       0.46***&      -0.75   \\
                &     (2.55)   &     (1.16)   &     (0.24)   &     (0.24)   &     (1.62)   &     (1.20)   &     (0.17)   &     (2.14)   \\
\addlinespace
1960            &       2.96   &      -1.92** &      -0.33   &      -0.27   &      -8.50***&      -1.02   &      -0.03   &       0.41   \\
                &     (2.13)   &     (0.88)   &     (0.23)   &     (0.23)   &     (1.53)   &     (0.85)   &     (0.13)   &     (2.32)   \\
\midrule
Observations    &     714.00   &     714.00   &     714.00   &     714.00   &     714.00   &     714.00   &     714.00   &     714.00   \\
\bottomrule
\multicolumn{9}{l}{\footnotesize Standard errors in parentheses}\\
\multicolumn{9}{l}{\footnotesize * p<0.10, ** p<0.05, *** p<0.01}\\
\end{tabular}
}

\clearpage
\end{landscape}

\section{County Comparisons: Overall}

This compares a set of "Treated" and "Control" counties in aggregate. The set is constructed by creating deciles of the county-level recent (1935-39) black migrant share of the 1940 population, then taking the counties with the largest (treated) and smallest (control) 1940-50 change in black population from each decile-census region (10 deciles x 3 regions = 30 possible pairs = 60 possible counties).The 1st and 10th deciles are dropped as the bins are constructed by frequencies, not values, thus their 1940 v2\_blackmig3539\_share values may not actually be that close. Doing this leaves us with 24 pairs/48 counties.

 These tables give summary statistics by decade on the endogenous X variable (GM), instrument (GM\_hat2), control variables (v2\_blackmig3539\_share and mfg\_lfshare), base values of outcome variables (base\_all\_local and base\_schdist\_ind), values of outcome variables (new\_all\_local and new\_schdist\_ind), and county population (countypop).

\foreach \var in {GM, GM_hat2, v2_blackmig3539_share, mfg_lfshare, base_all_local, base_schdist_ind, new_all_local, new_schdist_ind, countypop}{
	\input{tables/comparison_counties/overall_\var.tex}
}
\clearpage
\section{County Comparisons: By Region and blackmig3539\_share bins}
This section compares summary statistics for the 24 decile-census region pairs individually. 
\foreach \reg in {Northeast, Midwest, West}{
	\subsection{\reg Region}
	\foreach \bin in {1,2,3,4,5,6,7,8}{
		\input{tables/comparison_counties/byregXbin_\reg_bin_\bin.tex}
	}
	\clearpage
}

\end{document}