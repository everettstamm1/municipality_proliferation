\documentclass{article}
\usepackage{blindtext}
\usepackage{booktabs}
\usepackage[margin=0.25in]{geometry}
\usepackage{subcaption}
\usepackage{graphicx}
\usepackage{caption}
\usepackage{hyperref}
\usepackage{pdflscape}
\usepackage{tikz}
\usepackage{threeparttable}
\usepackage{algorithmic}


\title{Simple Tables for Municipality Proliferation}

\begin{document}
\maketitle
\tableofcontents
{\footnotesize 
\listoffigures
\listoftables}
\clearpage
\section{MAIN}

\begin{table}[htbp]\centering \def\sym#1{\ifmmode^{#1}\else\(^{#1}\)\fi}  \begin{threeparttable} \caption{Effects of change in Black Migration on Municipal Proliferation}
 \begin{tabular}{l*{8}{c}} \toprule
&\multicolumn{1}{c}{C. Goodman}&\multicolumn{4}{c}{Census of Governments}&\multicolumn{1}{c}{Census}\\\cmidrule(lr){2-2}\cmidrule(lr){3-6}\cmidrule(lr){7-7}
&\multicolumn{2}{c}{Municipalities}&\multicolumn{1}{c}{School districts}&\multicolumn{1}{c}{Townships}&\multicolumn{1}{c}{Special districts}&\multicolumn{1}{c}{Main City Share}\\\cmidrule(lr){2-3}\cmidrule(lr){4-6}\cmidrule(lr){7-7}
&\multicolumn{1}{c}{(1)}&\multicolumn{1}{c}{(2)}&\multicolumn{1}{c}{(3)}&\multicolumn{1}{c}{(4)}&\multicolumn{1}{c}{(5)}&\multicolumn{1}{c}{(6)}\\
\cmidrule(lr){1-7}
\multicolumn{6}{l}{Panel A: First Stage}\\
\cmidrule(lr){1-7}
$\widehat{GM}$  &    1.399***&    1.399***&    1.399***&    1.399***&    1.399***&    1.399***\\
                &  (0.233)   &  (0.233)   &  (0.233)   &  (0.233)   &  (0.233)   &  (0.233)   \\
\cmidrule(lr){1-7}
\multicolumn{6}{l}{Panel B: OLS}\\
\cmidrule(lr){1-7}
GM              &    0.004   &    0.008** &    0.304***&    0.017***&   -0.027***&   -1.099***\\
                &  (0.003)   &  (0.004)   &  (0.091)   &  (0.006)   &  (0.009)   &  (0.147)   \\
\cmidrule(lr){1-7}
\multicolumn{6}{l}{Panel C: Reduced Form}\\
\cmidrule(lr){1-7}
$\widehat{GM}$  &    0.010*  &    0.015** &    0.306   &    0.035***&   -0.010   &   -1.946***\\
                &  (0.006)   &  (0.007)   &  (0.194)   &  (0.011)   &  (0.015)   &  (0.293)   \\
\cmidrule(lr){1-7}
\multicolumn{6}{l}{Panel D: 2SLS}\\
\cmidrule(lr){1-7}
GM              &    0.007*  &    0.011** &    0.219*  &    0.025***&   -0.007   &   -1.391***\\
                &  (0.004)   &  (0.005)   &  (0.116)   &  (0.007)   &  (0.010)   &  (0.133)   \\
\midrule
First Stage F-Stat&    35.97   &    35.97   &    35.97   &    35.97   &    35.97   &    35.97   \\
Dep. Var. Mean  &    -0.14   &    -0.18   &    -3.64   &    -0.25   &     0.26   &   -14.58   \\
1940 Dep. Var. Mean&     0.63   &     0.68   &     4.10   &     0.81   &     0.43   &    50.06   \\
Observations    &      130   &      130   &      130   &      130   &      130   &      130   \\
       \bottomrule \end{tabular}

{\caption*{\begin{scriptsize} "\(p<0.10\), ** \(p<0.05\), *** \(p<0.01\)"\end{scriptsize}}} \end{threeparttable} \end{table}


\begin{table}[htbp]\centering \def\sym#1{\ifmmode^{#1}\else\(^{#1}\)\fi}  \begin{threeparttable} \caption{Effects of change in Black Migration on Municipal Proliferation, Percentage}
 \begin{tabular}{l*{7}{c}} \toprule
&\multicolumn{1}{c}{C. Goodman}&\multicolumn{4}{c}{Census of Governments}\\\cmidrule(lr){2-2}\cmidrule(lr){3-6}
&\multicolumn{2}{c}{Municipalities}&\multicolumn{1}{c}{School districts}&\multicolumn{1}{c}{Townships}&\multicolumn{1}{c}{Special districts}\\\cmidrule(lr){2-3}\cmidrule(lr){4-6}
&\multicolumn{1}{c}{(1)}&\multicolumn{1}{c}{(2)}&\multicolumn{1}{c}{(3)}&\multicolumn{1}{c}{(4)}&\multicolumn{1}{c}{(5)}\\
\cmidrule(lr){1-6}
\multicolumn{5}{l}{Panel A: First Stage}\\
\cmidrule(lr){1-6}
$\widehat{GM}$  &     1.26***&     1.26***&     1.26***&     1.26***&     1.26***\\
                &   (0.46)   &   (0.46)   &   (0.46)   &   (0.46)   &   (0.46)   \\
\cmidrule(lr){1-6}
\multicolumn{5}{l}{Panel B: OLS}\\
\cmidrule(lr){1-6}
GM              &    -0.01***&    -0.01** &     0.29***&     0.00   &    -0.04** \\
                &   (0.00)   &   (0.00)   &   (0.09)   &   (0.00)   &   (0.02)   \\
\cmidrule(lr){1-6}
\multicolumn{5}{l}{Panel C: Reduced Form}\\
\cmidrule(lr){1-6}
$\widehat{GM}$  &    -0.01   &     0.00   &     0.80** &    -0.00   &    -0.03   \\
                &   (0.01)   &   (0.01)   &   (0.33)   &   (0.01)   &   (0.05)   \\
\cmidrule(lr){1-6}
\multicolumn{5}{l}{Panel D: 2SLS}\\
\cmidrule(lr){1-6}
GM              &    -0.01   &     0.00   &     0.63***&    -0.00   &    -0.02   \\
                &   (0.01)   &   (0.01)   &   (0.23)   &   (0.00)   &   (0.04)   \\
\midrule
First Stage F-Stat&     7.35   &     7.35   &     7.35   &     7.35   &     7.35   \\
Dependent Variable Mean&      .11   &      .09   &    -3.32   &     -.02   &      .68   \\
Observations    &      130   &      130   &      130   &      130   &      130   \\
       \bottomrule \end{tabular}

{\caption*{\begin{scriptsize} "\(p<0.10\), ** \(p<0.05\), *** \(p<0.01\)"\end{scriptsize}}} \end{threeparttable} \end{table}


\clearpage
\begin{table}[htbp]\centering \def\sym#1{\ifmmode^{#1}\else\(^{#1}\)\fi}  \begin{threeparttable} \caption{Effects of change in Black Migration on Municipal Proliferation}
 \begin{tabular}{l*{8}{c}} \toprule
&\multicolumn{1}{c}{C. Goodman}&\multicolumn{4}{c}{Census of Governments}&\multicolumn{1}{c}{Census}\\\cmidrule(lr){2-2}\cmidrule(lr){3-6}\cmidrule(lr){7-7}
&\multicolumn{2}{c}{Municipalities}&\multicolumn{1}{c}{School districts}&\multicolumn{1}{c}{Townships}&\multicolumn{1}{c}{Special districts}&\multicolumn{1}{c}{Main City Share}\\\cmidrule(lr){2-3}\cmidrule(lr){4-6}\cmidrule(lr){7-7}
&\multicolumn{1}{c}{(1)}&\multicolumn{1}{c}{(2)}&\multicolumn{1}{c}{(3)}&\multicolumn{1}{c}{(4)}&\multicolumn{1}{c}{(5)}&\multicolumn{1}{c}{(6)}\\
\cmidrule(lr){1-7}
\multicolumn{6}{l}{Panel A: First Stage}\\
\cmidrule(lr){1-7}
$\widehat{GM}$  &    3.464***&    3.464***&    3.464***&    3.464***&    3.464***&    3.464***\\
                &  (0.418)   &  (0.418)   &  (0.418)   &  (0.418)   &  (0.418)   &  (0.418)   \\
\cmidrule(lr){1-7}
\multicolumn{6}{l}{Panel B: OLS}\\
\cmidrule(lr){1-7}
GM              &    0.003   &    0.005** &    0.181***&    0.011***&   -0.017** &   -0.800***\\
                &  (0.002)   &  (0.002)   &  (0.050)   &  (0.003)   &  (0.007)   &  (0.135)   \\
\cmidrule(lr){1-7}
\multicolumn{6}{l}{Panel C: Reduced Form}\\
\cmidrule(lr){1-7}
$\widehat{GM}$  &    0.019** &    0.027** &    0.918***&    0.067***&   -0.055** &   -4.145***\\
                &  (0.010)   &  (0.012)   &  (0.223)   &  (0.017)   &  (0.024)   &  (0.637)   \\
\cmidrule(lr){1-7}
\multicolumn{6}{l}{Panel D: 2SLS}\\
\cmidrule(lr){1-7}
GM              &    0.006** &    0.008***&    0.265***&    0.019***&   -0.016** &   -1.197***\\
                &  (0.002)   &  (0.003)   &  (0.061)   &  (0.004)   &  (0.006)   &  (0.138)   \\
\midrule
First Stage F-Stat&    68.63   &    68.63   &    68.63   &    68.63   &    68.63   &    68.63   \\
Dep. Var. Mean  &    -0.09   &    -0.10   &    -1.87   &    -0.16   &     0.19   &   -11.49   \\
1940 Dep. Var. Mean&     0.63   &     0.68   &     4.03   &     0.81   &     0.42   &    50.41   \\
Observations    &      130   &      130   &      130   &      130   &      130   &      130   \\
       \bottomrule \end{tabular}

{\caption*{\begin{scriptsize} "\(p<0.10\), ** \(p<0.05\), *** \(p<0.01\)"\end{scriptsize}}} \end{threeparttable} \end{table}


\clearpage
\begin{table}[htbp]\centering \def\sym#1{\ifmmode^{#1}\else\(^{#1}\)\fi}  \begin{threeparttable} \caption{Effects of change in Black Migration on Municipal Proliferation}
 \begin{tabular}{l*{7}{c}} \toprule
&\multicolumn{1}{c}{C. Goodman}&\multicolumn{4}{c}{Census of Governments}\\\cmidrule(lr){2-2}\cmidrule(lr){3-6}
&\multicolumn{2}{c}{Municipalities}&\multicolumn{1}{c}{School districts}&\multicolumn{1}{c}{Townships}&\multicolumn{1}{c}{Special districts}\\\cmidrule(lr){2-3}\cmidrule(lr){4-6}
&\multicolumn{1}{c}{(1)}&\multicolumn{1}{c}{(2)}&\multicolumn{1}{c}{(3)}&\multicolumn{1}{c}{(4)}&\multicolumn{1}{c}{(5)}\\
\cmidrule(lr){1-6}
\multicolumn{5}{l}{Panel A: First Stage}\\
\cmidrule(lr){1-6}
$\widehat{GM}$  &    3.464***&    3.464***&    3.464***&    3.464***&    3.464***\\
                &  (0.418)   &  (0.418)   &  (0.418)   &  (0.418)   &  (0.418)   \\
\cmidrule(lr){1-6}
\multicolumn{5}{l}{Panel B: OLS}\\
\cmidrule(lr){1-6}
GM              &    0.006   &   -0.005   &    0.020   &    0.031   &    0.091   \\
                &  (0.016)   &  (0.020)   &  (0.021)   &  (0.055)   &  (0.185)   \\
\cmidrule(lr){1-6}
\multicolumn{5}{l}{Panel C: Reduced Form}\\
\cmidrule(lr){1-6}
$\widehat{GM}$  &    0.031   &    0.002   &    0.117*  &    0.440***&    0.921   \\
                &  (0.056)   &  (0.064)   &  (0.069)   &  (0.146)   &  (0.859)   \\
\cmidrule(lr){1-6}
\multicolumn{5}{l}{Panel D: 2SLS}\\
\cmidrule(lr){1-6}
GM              &    0.009   &    0.000   &    0.035*  &    0.112***&    0.267   \\
                &  (0.016)   &  (0.018)   &  (0.020)   &  (0.030)   &  (0.244)   \\
\midrule
First Stage F-Stat&    68.63   &    68.63   &    68.63   &    68.63   &    68.63   \\
Dependent Variable Mean&     1.29   &     1.42   &     2.15   &     2.28   &    -3.45   \\
Observations    &      130   &      130   &      123   &       98   &      114   \\
       \bottomrule \end{tabular}

{\caption*{\begin{scriptsize} "\(p<0.10\), ** \(p<0.05\), *** \(p<0.01\)"\end{scriptsize}}} \end{threeparttable} \end{table}

\clearpage
\begin{table}[ht]
\centering
\caption{\textbf{Robustness of Effects on Municipalities to the Inclusion of Baseline Controls}}
\begin{threeparttable}
\begin{table}[htbp]\centering
\def\sym#1{\ifmmode^{#1}\else\(^{#1}\)\fi}
\caption{Outcome: cgoodman, }
\begin{tabular}{l*{10}{c}}
\toprule
            &\multicolumn{1}{c}{(1)}   &\multicolumn{1}{c}{(2)}   &\multicolumn{1}{c}{(3)}   &\multicolumn{1}{c}{(4)}   &\multicolumn{1}{c}{(5)}   &\multicolumn{1}{c}{(6)}   &\multicolumn{1}{c}{(7)}   &\multicolumn{1}{c}{(8)}   &\multicolumn{1}{c}{(9)}   &\multicolumn{1}{c}{(10)}   \\
\midrule
GM\_raw\_pp   &    0.0191** &    0.0297** &    0.0448***&    0.0423***&    0.0500***&    0.0497***&    0.0563***&    0.0531***&    0.0536***&    0.0660***\\
            & (0.00917)   &  (0.0133)   &  (0.0137)   &  (0.0152)   &  (0.0150)   &  (0.0165)   &  (0.0176)   &  (0.0159)   &  (0.0165)   &  (0.0240)   \\
\midrule
First stage F-Stat&    117.57   &96.38800000000001   &    68.633   &    56.256   &    57.904   &    49.437   &    36.905   &     56.28   &    56.768   &    33.802   \\
GM (OLS)    &       .02   &      .024   &      .027   &       .02   &      .028   &      .028   &       .03   &       .03   &       .03   &      .018   \\
R2 (OLS)    &      .051   &       .09   &      .092   &      .105   &      .102   &      .093   &      .095   &      .102   &      .099   &      .153   \\
Observations&       130   &       130   &       130   &       130   &       130   &       130   &       130   &       130   &       130   &       130   \\
Census Regions&         N   &         Y   &         Y   &         Y   &         Y   &         Y   &         Y   &         Y   &         Y   &         Y   \\
blackmig3539\_share&         N   &         N   &         Y   &         Y   &         Y   &         Y   &         Y   &         Y   &         Y   &         Y   \\
mfg\_lfshare &         N   &         N   &         N   &         Y   &         N   &         N   &         N   &         N   &         N   &         Y   \\
frac\_land   &         N   &         N   &         N   &         N   &         Y   &         N   &         N   &         N   &         N   &         Y   \\
totfrac\_in\_main\_city&         N   &         N   &         N   &         N   &         N   &         Y   &         N   &         N   &         N   &         Y   \\
m\_rr\_sqm2   &         N   &         N   &         N   &         N   &         N   &         N   &         Y   &         N   &         N   &         Y   \\
popc1940    &         N   &         N   &         N   &         N   &         N   &         N   &         N   &         Y   &         N   &         Y   \\
pop1940     &         N   &         N   &         N   &         N   &         N   &         N   &         N   &         N   &         Y   &         Y   \\
\bottomrule
\multicolumn{11}{l}{\footnotesize Standard errors in parentheses}\\
\multicolumn{11}{l}{\footnotesize * p<0.10, ** p<0.05, *** p<0.01}\\
\end{tabular}
\end{table}

\begin{tablenotes}\footnotesize
\item Column (3) of this table replicates Panel D Column (1) of asdfa. The remainder of the columns in the table alter specification choices to test for the stability of the point estimates  to the inclusion of various baseline controls... * \(p<0.10\), ** \(p<0.05\), *** \(p<0.01\)
\end{tablenotes}
\end{threeparttable}
\end{table}

\clearpage

\subsection{Alternative Instrument Tables}
\begin{landscape}
\begin{table}[ht]
\centering
\caption{\textbf{Robustness of Effects on Municipalities to Alternative Specifications}}
\begin{threeparttable}
\begin{table}[htbp]\centering
\def\sym#1{\ifmmode^{#1}\else\(^{#1}\)\fi}
\caption{Outcome: cgoodman, }
\begin{tabular}{l*{10}{c}}
\toprule
            &\multicolumn{1}{c}{(1)}   &\multicolumn{1}{c}{(2)}   &\multicolumn{1}{c}{(3)}   &\multicolumn{1}{c}{(4)}   &\multicolumn{1}{c}{(5)}   &\multicolumn{1}{c}{(6)}   &\multicolumn{1}{c}{(7)}   &\multicolumn{1}{c}{(8)}   &\multicolumn{1}{c}{(9)}   &\multicolumn{1}{c}{(10)}   \\
\midrule
GM\_raw\_pp   &    0.0191** &    0.0297** &    0.0448***&    0.0423***&    0.0500***&    0.0497***&    0.0563***&    0.0531***&    0.0536***&    0.0660***\\
            & (0.00917)   &  (0.0133)   &  (0.0137)   &  (0.0152)   &  (0.0150)   &  (0.0165)   &  (0.0176)   &  (0.0159)   &  (0.0165)   &  (0.0240)   \\
\midrule
First stage F-Stat&    117.57   &96.38800000000001   &    68.633   &    56.256   &    57.904   &    49.437   &    36.905   &     56.28   &    56.768   &    33.802   \\
GM (OLS)    &       .02   &      .024   &      .027   &       .02   &      .028   &      .028   &       .03   &       .03   &       .03   &      .018   \\
R2 (OLS)    &      .051   &       .09   &      .092   &      .105   &      .102   &      .093   &      .095   &      .102   &      .099   &      .153   \\
Observations&       130   &       130   &       130   &       130   &       130   &       130   &       130   &       130   &       130   &       130   \\
Census Regions&         N   &         Y   &         Y   &         Y   &         Y   &         Y   &         Y   &         Y   &         Y   &         Y   \\
blackmig3539\_share&         N   &         N   &         Y   &         Y   &         Y   &         Y   &         Y   &         Y   &         Y   &         Y   \\
mfg\_lfshare &         N   &         N   &         N   &         Y   &         N   &         N   &         N   &         N   &         N   &         Y   \\
frac\_land   &         N   &         N   &         N   &         N   &         Y   &         N   &         N   &         N   &         N   &         Y   \\
totfrac\_in\_main\_city&         N   &         N   &         N   &         N   &         N   &         Y   &         N   &         N   &         N   &         Y   \\
m\_rr\_sqm2   &         N   &         N   &         N   &         N   &         N   &         N   &         Y   &         N   &         N   &         Y   \\
popc1940    &         N   &         N   &         N   &         N   &         N   &         N   &         N   &         Y   &         N   &         Y   \\
pop1940     &         N   &         N   &         N   &         N   &         N   &         N   &         N   &         N   &         Y   &         Y   \\
\bottomrule
\multicolumn{11}{l}{\footnotesize Standard errors in parentheses}\\
\multicolumn{11}{l}{\footnotesize * p<0.10, ** p<0.05, *** p<0.01}\\
\end{tabular}
\end{table}

\begin{tablenotes}\footnotesize
\item Column (3) adjusts the outcome variable by total population, rather than urban population. Columns (4), (5), (6), and (7) are th: Column (4) uses an instrument residualized on southern state fixed effects. This accounts for shocks correlated between southern states and non-southern destinations. Column (5) drops the 15 southern counties coded as central in MSAs with a 1990 population over one million before constructing the instrument. This accounts for shocks correlated across both southern and non-southern urban areas. Column (6) constructs the migration links using southern state of birth of recent black migrants. Column (7) uses southern white migrants as the instrument and endogeneous variable to validate that this phenomenon is regarding Black southern migrants, not just any southern migrants. Columns (8), (9), (10), and (11) use the 1940 full count census from IPUMS [cite ipums], rather than the intermediate/cleaned version used in , to construct the destination sample, which allows us to allow us to modify the sample in two important ways. Column (8) validates the use of this sample, the specification is otherwise equivalent to column (1). Column (9) switches Texas from a southern to a non-southern city. Column (10) uses rural migrants only, defined as having reported moving from outside of an incorporated city between 1935-40. Column (11) employs both northern Texas and rural migrants only.  * \(p<0.10\), ** \(p<0.05\), *** \(p<0.01\)
\end{tablenotes}
\end{threeparttable}
\label{tab:cgoodman_insts}
\end{table}
\clearpage

\begin{table}[ht]
\centering
\caption{\textbf{Robustness of Effects on Municipalities to Alternative Specifications}}
\begin{threeparttable}
 \begin{tabular}{l*{15}{c}} \toprule
                    &\multicolumn{1}{c}{(1)}   &\multicolumn{1}{c}{(2)}   &\multicolumn{1}{c}{(3)}   &\multicolumn{1}{c}{(4)}   &\multicolumn{1}{c}{(5)}   &\multicolumn{1}{c}{(6)}   &\multicolumn{1}{c}{(7)}   &\multicolumn{1}{c}{(8)}   &\multicolumn{1}{c}{(9)}   &\multicolumn{1}{c}{(10)}   &\multicolumn{1}{c}{(11)}   &\multicolumn{1}{c}{(12)}   &\multicolumn{1}{c}{(13)}   \\
\midrule
Percentage Point Change in Urban Black Population& -0.01   & -0.00   &  0.02** &  0.01*  &  0.01   &  0.02** &  0.01   &  0.01   &  0.01*  &  0.00   &  0.00   &  0.02***&  0.01   \\
                    &(0.01)   &(0.01)   &(0.01)   &(0.01)   &(0.01)   &(0.01)   &(0.01)   &(0.01)   &(0.01)   &(0.01)   &(0.01)   &(0.01)   &(0.01)   \\
\midrule
First stage F-Stat  &117.57   & 96.39   & 68.63   & 57.90   & 49.44   & 59.90   & 56.28   & 56.77   & 56.26   & 53.87   & 40.50   & 55.55   & 24.02   \\
GM (OLS)            & -0.01   &  0.00   &  0.01   &  0.01   &  0.01   &  0.01   &  0.01   &  0.00   &  0.01   & -0.00   & -0.00   &  0.01   & -0.01   \\
R2 (OLS)            &  0.03   &  0.23   &  0.34   &  0.39   &  0.41   &  0.34   &  0.41   &  0.44   &  0.35   &  0.61   &  0.46   &  0.36   &  0.74   \\
Observations        &   130   &   130   &   130   &   130   &   130   &   130   &   130   &   130   &   130   &   130   &   130   &   130   &   130   \\
Census region FEs   &     N   &     Y   &     Y   &     Y   &     Y   &     Y   &     Y   &     Y   &     Y   &     Y   &     Y   &     Y   &     Y   \\
Fraction of recent southern Black migrants&     N   &     N   &     Y   &     Y   &     Y   &     Y   &     Y   &     Y   &     Y   &     Y   &     Y   &     Y   &     Y   \\
Fraction of land incorporated, 1940&     N   &     N   &     N   &     Y   &     N   &     N   &     N   &     N   &     N   &     N   &     N   &     N   &     Y   \\
Fraction of CZ population in largest city&     N   &     N   &     N   &     N   &     Y   &     N   &     N   &     N   &     N   &     N   &     N   &     N   &     Y   \\
Meters of railroad per square meter of land&     N   &     N   &     N   &     N   &     N   &     Y   &     N   &     N   &     N   &     N   &     N   &     N   &     Y   \\
1940 urban population&     N   &     N   &     N   &     N   &     N   &     N   &     Y   &     N   &     N   &     N   &     N   &     N   &     Y   \\
1940 total population&     N   &     N   &     N   &     N   &     N   &     N   &     N   &     Y   &     N   &     N   &     N   &     N   &     Y   \\
1940 manufacturing share&     N   &     N   &     N   &     N   &     N   &     N   &     N   &     N   &     Y   &     N   &     N   &     N   &     Y   \\
1940 baseline outcome&     N   &     N   &     N   &     N   &     N   &     N   &     N   &     N   &     N   &     Y   &     N   &     N   &     Y   \\
Log 1940 population density&     N   &     N   &     N   &     N   &     N   &     N   &     N   &     N   &     N   &     N   &     Y   &     N   &     Y   \\
1940 urban fraction &     N   &     N   &     N   &     N   &     N   &     N   &     N   &     N   &     N   &     N   &     N   &     Y   &     Y   \\
\bottomrule \end{tabular}

\begin{tablenotes}\footnotesize
\item Column (3) adjusts the outcome variable by total population, rather than urban population. Columns (4), (5), (6), and (7) are th: Column (4) uses an instrument residualized on southern state fixed effects. This accounts for shocks correlated between southern states and non-southern destinations. Column (5) drops the 15 southern counties coded as central in MSAs with a 1990 population over one million before constructing the instrument. This accounts for shocks correlated across both southern and non-southern urban areas. Column (6) constructs the migration links using southern state of birth of recent black migrants. Column (7) uses southern white migrants as the instrument and endogeneous variable to validate that this phenomenon is regarding Black southern migrants, not just any southern migrants. Columns (8), (9), (10), and (11) use the 1940 full count census from IPUMS [cite ipums], rather than the intermediate/cleaned version used in , to construct the destination sample, which allows us to allow us to modify the sample in two important ways. Column (8) validates the use of this sample, the specification is otherwise equivalent to column (1). Column (9) switches Texas from a southern to a non-southern city. Column (10) uses rural migrants only, defined as having reported moving from outside of an incorporated city between 1935-40. Column (11) employs both northern Texas and rural migrants only.  * \(p<0.10\), ** \(p<0.05\), *** \(p<0.01\)
\end{tablenotes}
\end{threeparttable}
\label{tab:gen_muni_insts}
\end{table}
\clearpage


\begin{table}[ht]
\centering
\caption{\textbf{Robustness of Effects on School Districts to Alternative Specifications}}
\begin{threeparttable}
 \begin{tabular}{l*{15}{c}} \toprule
                    &\multicolumn{1}{c}{(1)}   &\multicolumn{1}{c}{(2)}   &\multicolumn{1}{c}{(3)}   &\multicolumn{1}{c}{(4)}   &\multicolumn{1}{c}{(5)}   &\multicolumn{1}{c}{(6)}   &\multicolumn{1}{c}{(7)}   &\multicolumn{1}{c}{(8)}   &\multicolumn{1}{c}{(9)}   &\multicolumn{1}{c}{(10)}   &\multicolumn{1}{c}{(11)}   &\multicolumn{1}{c}{(12)}   &\multicolumn{1}{c}{(13)}   \\
\midrule
Percentage Point Change in Urban Black Population&  0.20***&  0.47***&  0.42***&  0.38***&  0.38***&  0.38***&  0.38***&  0.37***&  0.30** & -0.00   &  0.21*  &  0.39***&  0.01   \\
                    &(0.07)   &(0.09)   &(0.12)   &(0.11)   &(0.12)   &(0.12)   &(0.11)   &(0.12)   &(0.13)   &(0.01)   &(0.12)   &(0.13)   &(0.01)   \\
\midrule
First stage F-Stat  &117.57   & 96.39   & 68.63   & 57.90   & 49.44   & 59.90   & 56.28   & 56.77   & 56.26   & 53.11   & 40.50   & 55.55   & 27.02   \\
GM (OLS)            &  0.26   &  0.35   &  0.29   &  0.27   &  0.26   &  0.23   &  0.26   &  0.24   &  0.15   & -0.00   &  0.09   &  0.25   &  0.01   \\
R2 (OLS)            &  0.12   &  0.35   &  0.36   &  0.38   &  0.38   &  0.39   &  0.38   &  0.38   &  0.44   &  1.00   &  0.47   &  0.38   &  1.00   \\
Observations        &   130   &   130   &   130   &   130   &   130   &   130   &   130   &   130   &   130   &   130   &   130   &   130   &   130   \\
Census region FEs   &     N   &     Y   &     Y   &     Y   &     Y   &     Y   &     Y   &     Y   &     Y   &     Y   &     Y   &     Y   &     Y   \\
Fraction of recent southern Black migrants&     N   &     N   &     Y   &     Y   &     Y   &     Y   &     Y   &     Y   &     Y   &     Y   &     Y   &     Y   &     Y   \\
Fraction of land incorporated, 1940&     N   &     N   &     N   &     Y   &     N   &     N   &     N   &     N   &     N   &     N   &     N   &     N   &     Y   \\
Fraction of CZ population in largest city&     N   &     N   &     N   &     N   &     Y   &     N   &     N   &     N   &     N   &     N   &     N   &     N   &     Y   \\
Meters of railroad per square meter of land&     N   &     N   &     N   &     N   &     N   &     Y   &     N   &     N   &     N   &     N   &     N   &     N   &     Y   \\
1940 urban population&     N   &     N   &     N   &     N   &     N   &     N   &     Y   &     N   &     N   &     N   &     N   &     N   &     Y   \\
1940 total population&     N   &     N   &     N   &     N   &     N   &     N   &     N   &     Y   &     N   &     N   &     N   &     N   &     Y   \\
1940 manufacturing share&     N   &     N   &     N   &     N   &     N   &     N   &     N   &     N   &     Y   &     N   &     N   &     N   &     Y   \\
1940 baseline outcome&     N   &     N   &     N   &     N   &     N   &     N   &     N   &     N   &     N   &     Y   &     N   &     N   &     Y   \\
Log 1940 population density&     N   &     N   &     N   &     N   &     N   &     N   &     N   &     N   &     N   &     N   &     Y   &     N   &     Y   \\
1940 urban fraction &     N   &     N   &     N   &     N   &     N   &     N   &     N   &     N   &     N   &     N   &     N   &     Y   &     Y   \\
\bottomrule \end{tabular}

\begin{tablenotes}\footnotesize
\item Column (3) adjusts the outcome variable by total population, rather than urban population. Columns (4), (5), (6), and (7) are th: Column (4) uses an instrument residualized on southern state fixed effects. This accounts for shocks correlated between southern states and non-southern destinations. Column (5) drops the 15 southern counties coded as central in MSAs with a 1990 population over one million before constructing the instrument. This accounts for shocks correlated across both southern and non-southern urban areas. Column (6) constructs the migration links using southern state of birth of recent black migrants. Column (7) uses southern white migrants as the instrument and endogeneous variable to validate that this phenomenon is regarding Black southern migrants, not just any southern migrants. Columns (8), (9), (10), and (11) use the 1940 full count census from IPUMS [cite ipums], rather than the intermediate/cleaned version used in , to construct the destination sample, which allows us to allow us to modify the sample in two important ways. Column (8) validates the use of this sample, the specification is otherwise equivalent to column (1). Column (9) switches Texas from a southern to a non-southern city. Column (10) uses rural migrants only, defined as having reported moving from outside of an incorporated city between 1935-40. Column (11) employs both northern Texas and rural migrants only.  * \(p<0.10\), ** \(p<0.05\), *** \(p<0.01\)
\end{tablenotes}
\end{threeparttable}
\label{tab:schdist_ind_insts}
\end{table}

\begin{table}[ht]
\centering
\caption{\textbf{Robustness of Effects on Townships to Alternative Specifications}}
\begin{threeparttable}
 \begin{tabular}{l*{15}{c}} \toprule
                    &\multicolumn{1}{c}{(1)}   &\multicolumn{1}{c}{(2)}   &\multicolumn{1}{c}{(3)}   &\multicolumn{1}{c}{(4)}   &\multicolumn{1}{c}{(5)}   &\multicolumn{1}{c}{(6)}   &\multicolumn{1}{c}{(7)}   &\multicolumn{1}{c}{(8)}   &\multicolumn{1}{c}{(9)}   &\multicolumn{1}{c}{(10)}   &\multicolumn{1}{c}{(11)}   &\multicolumn{1}{c}{(12)}   &\multicolumn{1}{c}{(13)}   \\
\midrule
Percentage Point Change in Urban Black Population&  0.20***&  0.47***&  0.42***&  0.38***&  0.38***&  0.38***&  0.38***&  0.37***&  0.30** & -0.00   &  0.21*  &  0.39***&  0.01   \\
                    &(0.07)   &(0.09)   &(0.12)   &(0.11)   &(0.12)   &(0.12)   &(0.11)   &(0.12)   &(0.13)   &(0.01)   &(0.12)   &(0.13)   &(0.01)   \\
\midrule
First stage F-Stat  &117.57   & 96.39   & 68.63   & 57.90   & 49.44   & 59.90   & 56.28   & 56.77   & 56.26   & 53.11   & 40.50   & 55.55   & 27.02   \\
GM (OLS)            &  0.26   &  0.35   &  0.29   &  0.27   &  0.26   &  0.23   &  0.26   &  0.24   &  0.15   & -0.00   &  0.09   &  0.25   &  0.01   \\
R2 (OLS)            &  0.12   &  0.35   &  0.36   &  0.38   &  0.38   &  0.39   &  0.38   &  0.38   &  0.44   &  1.00   &  0.47   &  0.38   &  1.00   \\
Observations        &   130   &   130   &   130   &   130   &   130   &   130   &   130   &   130   &   130   &   130   &   130   &   130   &   130   \\
Census region FEs   &     N   &     Y   &     Y   &     Y   &     Y   &     Y   &     Y   &     Y   &     Y   &     Y   &     Y   &     Y   &     Y   \\
Fraction of recent southern Black migrants&     N   &     N   &     Y   &     Y   &     Y   &     Y   &     Y   &     Y   &     Y   &     Y   &     Y   &     Y   &     Y   \\
Fraction of land incorporated, 1940&     N   &     N   &     N   &     Y   &     N   &     N   &     N   &     N   &     N   &     N   &     N   &     N   &     Y   \\
Fraction of CZ population in largest city&     N   &     N   &     N   &     N   &     Y   &     N   &     N   &     N   &     N   &     N   &     N   &     N   &     Y   \\
Meters of railroad per square meter of land&     N   &     N   &     N   &     N   &     N   &     Y   &     N   &     N   &     N   &     N   &     N   &     N   &     Y   \\
1940 urban population&     N   &     N   &     N   &     N   &     N   &     N   &     Y   &     N   &     N   &     N   &     N   &     N   &     Y   \\
1940 total population&     N   &     N   &     N   &     N   &     N   &     N   &     N   &     Y   &     N   &     N   &     N   &     N   &     Y   \\
1940 manufacturing share&     N   &     N   &     N   &     N   &     N   &     N   &     N   &     N   &     Y   &     N   &     N   &     N   &     Y   \\
1940 baseline outcome&     N   &     N   &     N   &     N   &     N   &     N   &     N   &     N   &     N   &     Y   &     N   &     N   &     Y   \\
Log 1940 population density&     N   &     N   &     N   &     N   &     N   &     N   &     N   &     N   &     N   &     N   &     Y   &     N   &     Y   \\
1940 urban fraction &     N   &     N   &     N   &     N   &     N   &     N   &     N   &     N   &     N   &     N   &     N   &     Y   &     Y   \\
\bottomrule \end{tabular}

\begin{tablenotes}\footnotesize
\item Column (3) adjusts the outcome variable by total population, rather than urban population. Columns (4), (5), (6), and (7) are th: Column (4) uses an instrument residualized on southern state fixed effects. This accounts for shocks correlated between southern states and non-southern destinations. Column (5) drops the 15 southern counties coded as central in MSAs with a 1990 population over one million before constructing the instrument. This accounts for shocks correlated across both southern and non-southern urban areas. Column (6) constructs the migration links using southern state of birth of recent black migrants. Column (7) uses southern white migrants as the instrument and endogeneous variable to validate that this phenomenon is regarding Black southern migrants, not just any southern migrants. Columns (8), (9), (10), and (11) use the 1940 full count census from IPUMS [cite ipums], rather than the intermediate/cleaned version used in , to construct the destination sample, which allows us to allow us to modify the sample in two important ways. Column (8) validates the use of this sample, the specification is otherwise equivalent to column (1). Column (9) switches Texas from a southern to a non-southern city. Column (10) uses rural migrants only, defined as having reported moving from outside of an incorporated city between 1935-40. Column (11) employs both northern Texas and rural migrants only.  * \(p<0.10\), ** \(p<0.05\), *** \(p<0.01\)
\end{tablenotes}
\end{threeparttable}
\label{tab:gen_town_insts}
\end{table}

\begin{table}[ht]
\centering
\caption{\textbf{Robustness of Effects on Special Districts to Alternative Specifications}}
\begin{threeparttable}
 \begin{tabular}{l*{15}{c}} \toprule
                    &\multicolumn{1}{c}{(1)}   &\multicolumn{1}{c}{(2)}   &\multicolumn{1}{c}{(3)}   &\multicolumn{1}{c}{(4)}   &\multicolumn{1}{c}{(5)}   &\multicolumn{1}{c}{(6)}   &\multicolumn{1}{c}{(7)}   &\multicolumn{1}{c}{(8)}   &\multicolumn{1}{c}{(9)}   &\multicolumn{1}{c}{(10)}   &\multicolumn{1}{c}{(11)}   &\multicolumn{1}{c}{(12)}   &\multicolumn{1}{c}{(13)}   \\
\midrule
Percentage Point Change in Urban Black Population& -0.02***& -0.04***& -0.02** & -0.01   & -0.02*  & -0.03***& -0.01   & -0.01   & -0.02*  & -0.02***& -0.00   & -0.01   &  0.00   \\
                    &(0.01)   &(0.01)   &(0.01)   &(0.01)   &(0.01)   &(0.01)   &(0.01)   &(0.01)   &(0.01)   &(0.01)   &(0.01)   &(0.01)   &(0.02)   \\
\midrule
First stage F-Stat  &117.57   & 96.39   & 68.63   & 57.90   & 49.44   & 59.90   & 56.28   & 56.77   & 56.26   & 72.34   & 40.50   & 55.55   & 21.19   \\
GM (OLS)            & -0.03   & -0.03   & -0.03   & -0.02   & -0.03   & -0.04   & -0.02   & -0.02   & -0.03   & -0.03   & -0.01   & -0.02   & -0.03   \\
R2 (OLS)            &  0.14   &  0.22   &  0.23   &  0.27   &  0.24   &  0.31   &  0.26   &  0.26   &  0.23   &  0.25   &  0.27   &  0.27   &  0.48   \\
Observations        &   130   &   130   &   130   &   130   &   130   &   130   &   130   &   130   &   130   &   130   &   130   &   130   &   130   \\
Census region FEs   &     N   &     Y   &     Y   &     Y   &     Y   &     Y   &     Y   &     Y   &     Y   &     Y   &     Y   &     Y   &     Y   \\
Fraction of recent southern Black migrants&     N   &     N   &     Y   &     Y   &     Y   &     Y   &     Y   &     Y   &     Y   &     Y   &     Y   &     Y   &     Y   \\
Fraction of land incorporated, 1940&     N   &     N   &     N   &     Y   &     N   &     N   &     N   &     N   &     N   &     N   &     N   &     N   &     Y   \\
Fraction of CZ population in largest city&     N   &     N   &     N   &     N   &     Y   &     N   &     N   &     N   &     N   &     N   &     N   &     N   &     Y   \\
Meters of railroad per square meter of land&     N   &     N   &     N   &     N   &     N   &     Y   &     N   &     N   &     N   &     N   &     N   &     N   &     Y   \\
1940 urban population&     N   &     N   &     N   &     N   &     N   &     N   &     Y   &     N   &     N   &     N   &     N   &     N   &     Y   \\
1940 total population&     N   &     N   &     N   &     N   &     N   &     N   &     N   &     Y   &     N   &     N   &     N   &     N   &     Y   \\
1940 manufacturing share&     N   &     N   &     N   &     N   &     N   &     N   &     N   &     N   &     Y   &     N   &     N   &     N   &     Y   \\
1940 baseline outcome&     N   &     N   &     N   &     N   &     N   &     N   &     N   &     N   &     N   &     Y   &     N   &     N   &     Y   \\
Log 1940 population density&     N   &     N   &     N   &     N   &     N   &     N   &     N   &     N   &     N   &     N   &     Y   &     N   &     Y   \\
1940 urban fraction &     N   &     N   &     N   &     N   &     N   &     N   &     N   &     N   &     N   &     N   &     N   &     Y   &     Y   \\
\bottomrule \end{tabular}

\begin{tablenotes}\footnotesize
\item Column (3) adjusts the outcome variable by total population, rather than urban population. Columns (4), (5), (6), and (7) are th: Column (4) uses an instrument residualized on southern state fixed effects. This accounts for shocks correlated between southern states and non-southern destinations. Column (5) drops the 15 southern counties coded as central in MSAs with a 1990 population over one million before constructing the instrument. This accounts for shocks correlated across both southern and non-southern urban areas. Column (6) constructs the migration links using southern state of birth of recent black migrants. Column (7) uses southern white migrants as the instrument and endogeneous variable to validate that this phenomenon is regarding Black southern migrants, not just any southern migrants. Columns (8), (9), (10), and (11) use the 1940 full count census from IPUMS [cite ipums], rather than the intermediate/cleaned version used in , to construct the destination sample, which allows us to allow us to modify the sample in two important ways. Column (8) validates the use of this sample, the specification is otherwise equivalent to column (1). Column (9) switches Texas from a southern to a non-southern city. Column (10) uses rural migrants only, defined as having reported moving from outside of an incorporated city between 1935-40. Column (11) employs both northern Texas and rural migrants only.  * \(p<0.10\), ** \(p<0.05\), *** \(p<0.01\)
\end{tablenotes}
\end{threeparttable}
\label{tab:spdist_insts}
\end{table}
\end{landscape}
\subsection{Balance Table}
{
\def\sym#1{\ifmmode^{#1}\else\(^{#1}\)\fi}
\begin{tabular}{l*{5}{c}}
\toprule
                &\multicolumn{1}{c}{1940-1970 Pooled}&\multicolumn{1}{c}{1940-1950}&\multicolumn{1}{c}{1950-1960}&\multicolumn{1}{c}{1960-1970}&\multicolumn{1}{c}{Stacked}\\
\midrule
mfg\_lfshare on GM\_hat&     0.56         &     1.60         &     0.39         &     1.34         &     1.15         \\
                &   (0.65)         &   (1.19)         &   (1.97)         &   (1.59)         &   (0.81)         \\
\addlinespace
blackmig3539 on GM\_hat&     0.06\sym{***}&     0.07\sym{*}  &     0.18\sym{***}&     0.14\sym{***}&     0.08\sym{***}\\
                &   (0.01)         &   (0.03)         &   (0.01)         &   (0.02)         &   (0.02)         \\
\addlinespace
frac\_land on GM\_hat&     0.04         &     0.06         &     0.16\sym{*}  &     0.16\sym{*}  &     0.08\sym{**} \\
                &   (0.02)         &   (0.03)         &   (0.08)         &   (0.08)         &   (0.03)         \\
\addlinespace
transpo\_cost\_1920 on GM\_hat&    -0.08\sym{*}  &    -0.14         &    -0.27\sym{*}  &    -0.24\sym{*}  &    -0.15\sym{**} \\
                &   (0.03)         &   (0.08)         &   (0.11)         &   (0.12)         &   (0.05)         \\
\addlinespace
coastal on GM\_hat&     0.03\sym{*}  &     0.02         &     0.11\sym{*}  &     0.11         &     0.05         \\
                &   (0.01)         &   (0.03)         &   (0.05)         &   (0.06)         &   (0.03)         \\
\addlinespace
avg\_precip on GM\_hat&     0.55         &     1.14         &     2.73         &    -0.06         &     0.97         \\
                &   (0.54)         &   (0.98)         &   (1.86)         &   (1.75)         &   (0.78)         \\
\addlinespace
avg\_temp on GM\_hat&    -1.27         &    -1.15         &    -2.54         &    -6.19         &    -2.05         \\
                &   (1.24)         &   (2.67)         &   (3.46)         &   (5.06)         &   (2.17)         \\
\addlinespace
n\_wells on GM\_hat&   -12.20         &   -20.12         &   -18.36         &   -71.76         &   -24.48\sym{*}  \\
                &   (7.01)         &  (13.21)         &  (19.05)         &  (42.38)         &  (11.77)         \\
\addlinespace
totfrac\_in\_main\_city on GM\_hat&     0.06\sym{**} &     0.08\sym{**} &     0.18\sym{**} &     0.19\sym{**} &     0.10\sym{***}\\
                &   (0.02)         &   (0.03)         &   (0.07)         &   (0.07)         &   (0.03)         \\
\addlinespace
urbfrac\_in\_main\_city on GM\_hat&     0.02         &     0.03         &     0.09         &     0.05         &     0.04\sym{*}  \\
                &   (0.02)         &   (0.02)         &   (0.05)         &   (0.05)         &   (0.02)         \\
\addlinespace
m\_rr on GM\_hat  &  1.2e+05\sym{*}  & 83064.97         &  3.1e+05         &  7.7e+05\sym{**} &  2.2e+05         \\
                &(52356.24)         &(99869.56)         &(1.8e+05)         &(2.7e+05)         &(1.3e+05)         \\
\addlinespace
m\_rr\_sqm2 on GM\_hat&     0.00         &     0.00         &     0.00         &     0.00         &     0.00\sym{*}  \\
                &   (0.00)         &   (0.00)         &   (0.00)         &   (0.00)         &   (0.00)         \\
\addlinespace
popc1940 on GM\_hat&  5.3e+05\sym{*}  &  6.6e+05\sym{*}  &  1.8e+06\sym{*}  &  2.0e+06\sym{**} &  9.6e+05\sym{***}\\
                &(2.1e+05)         &(3.0e+05)         &(7.0e+05)         &(7.5e+05)         &(2.7e+05)         \\
\addlinespace
pop1940 on GM\_hat&  5.9e+05\sym{**} &  7.1e+05\sym{*}  &  1.9e+06\sym{**} &  2.3e+06\sym{**} &  1.1e+06\sym{***}\\
                &(2.2e+05)         &(3.1e+05)         &(7.1e+05)         &(7.9e+05)         &(3.0e+05)         \\
\bottomrule
\multicolumn{6}{l}{\footnotesize Standard errors in parentheses}\\
\multicolumn{6}{l}{\footnotesize \sym{*} \(p<0.05\), \sym{**} \(p<0.01\), \sym{***} \(p<0.001\)}\\
\end{tabular}
}

\clearpage

\end{document}