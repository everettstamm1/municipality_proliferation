\begin{table}[htbp]\centering
\def\sym#1{\ifmmode^{#1}\else\(^{#1}\)\fi}
\caption{Regressing Number of New School Districts On School Finance Data}
\begin{tabular}{l*{6}{c}}
\toprule
                    &Expenditure Per Student   &               &               &Local Revenue Per Student   &               &               \\
                    &\multicolumn{1}{c}{(1)}   &\multicolumn{1}{c}{(2)}   &\multicolumn{1}{c}{(3)}   &\multicolumn{1}{c}{(4)}   &\multicolumn{1}{c}{(5)}   &\multicolumn{1}{c}{(6)}   \\
\midrule
Number of Local Govts&      -156.8***&      -127.8***&      -129.8***&      -52.64***&      -39.37***&      -41.04***\\
                    &     (25.65)   &     (26.50)   &     (27.03)   &     (10.64)   &     (11.04)   &     (11.58)   \\
\midrule
R-Squared           &        .147   &        .264   &        .271   &        .102   &         .21   &        .225   \\
Dep Var Mean        &       24000   &       24000   &       24000   &       10000   &       10000   &       10000   \\
Mfg/Black Mig Controls&          No   &         Yes   &         Yes   &          No   &         Yes   &         Yes   \\
TRI Controls        &          No   &          No   &         Yes   &          No   &          No   &         Yes   \\
Observations        &        1608   &        1608   &        1608   &        1608   &        1608   &        1608   \\
\bottomrule
\multicolumn{7}{l}{\footnotesize X variable is number of new school districts, Per Capita (100,000) per county by decade for 1940-50, 1950-60, and 1960-70. Y variable is county-level average Local Revenue per student from 1994-2018. Controls include base decade number of independent school districts and region and (X variable) decade fixed effects. Standard errors clustered at county level.}\\
\multicolumn{7}{l}{\footnotesize * p<0.10, ** p<0.05, *** p<0.01}\\
\end{tabular}
\end{table}
